% latexcad.tex
% Documentation file (LaTeX) for LaTeX-CAD
% John Leis leis@usq.edu.au

% Version 1.0 Orig 4 November 1995
% Version 1.1 1 Dec 1995, (1.1a 11 Dec)
% Version 1.2 14 Dec 1995
% Version 1.3 17 Dec 1995 
% Version 1.4 20 Dec 1995
% Version 1.5 27 January 1996
% Version 1.6  18 February 1996
%              21 March 1996 - modified to 2e compatibility
% Version 1.7 8 Aug 1996 - shadebox and shadowbox bugs fixed
% Version 1.8  Nov 1996 - archive with epic/eepic fixed
% Version 1.9  Mar 1998 - archive with epic/eepic finally fixed

\documentclass[a4paper]{article}

\usepackage{lgrind}     % for formated source code
\usepackage{latexcad}   % latexcad.sty for drawings etc


\newcommand{\bs}{$\backslash$}
\pagestyle{headings}

\begin{document}

\title{ \large\bf  \LaTeX-CAD - A Drawing Package for \LaTeXe\ }
\author{John W Leis \\ leis@usq.edu.au}
\date{Version 1.9 \\ March 1998}
\maketitle

%%%%% Introduction
\section{Introduction}


%% disclaimer
\subsection{Disclaimer}
The author accepts no liability, express or implied, for any 
damages, whether direct or consequential, associated with the
use of this software. 

%%%% availability
\subsection{Availability}
LaTeX-CAD is freely available and may be distributed provided
no files are modified, added to in any way, or deleted.
If you use this software, an email to the author would be appreciated.

%%%% scope
\subsection{Scope}
LaTeX-CAD is a drawing package for creating \LaTeX-compatible line
diagrams under the Windows environment. It can create a variety of
lines and shapes incorporating \LaTeX\/ fonts and symbols.
The package includes a number of macros for incorporating the
resulting drawings into a \LaTeX\/ source document. In addition to a
variety of illustrating shapes, the package may be used to annotate
plots produced by the GnuPlot plotting package. Figures~\ref{fig:lcadeg}
and~\ref{fig:gnuplot} show simple illustrations of these features.
LaTeX-CAD has been tested on screen resolutions down to $800\times 600$ SVGA.
Users with screen resolutions lower than this, including standard VGA, may
experience problems - this is due to the compromise between clarity at higher
resolutions and drawing size at lower resolutions. A higher resolution
(e.g. $1024\times 768$) is strongly recommended.
I believe LaTeX-CAD will now (Version 1.8+) operate satisfactorily under
Windows '95.

%%%%%% Capabilities
\subsection{ Capabilities }
A variety of standard line-drawing shapes are available in \LaTeX-CAD.
These are illustrated in Figure~\ref{fig:lcadeg}.
Section~\ref{sec:objecttypes} details these object types.

\placedrawing{lcadeg.lp}{\LaTeX-CAD objects}{fig:lcadeg}
\placedrawing{gnuplot.lp}{A \LaTeX-CAD example (interactive editing of a GnuPlot graph).}{fig:gnuplot}


%%%%%%% Requirements
\subsection{ Requirements }
In order to use \LaTeX-CAD, the following system requirements must be met:-
\begin{itemize}
	\item Windows 3.1x ;
	\item \LaTeX\/ package ;
	\item A DVI viewer capable of interpreting tpic \bs special's, and
	\item Metafont, for generating missing fonts. (Note: dviwin may
		be set up for automatic generation of missing fonts.)
\end{itemize}
The author uses the g\TeX\ distribution (available on CTAN archive sites),
which includes: -
\begin{itemize}
	\item g\TeX\ distribution 2.1
	\item \TeX\ 3.1415 (C Version 6.1) ;
	\item dviwin 2.9
	\item metafont 2.71
	\item dvips 5.5 (optional, for PostScript output only)
\end{itemize}

LaTeX-CAD 1.9 has been tested with with \LaTeXe\ .

%%%%%% Installation 
\section{ Installation }
The files shown in Table~\ref{tab:files} are contained in the
distribution.
A defaults file, latexcad.ini is also required to store
user's preferences. This is created by the application if
it does not already exist, and will be located in
the \bs windows directory.

\begin{table}
	\begin{tabular}{lll}
		\hline\hline
		File Name & Description & Installed Location \\
		\hline
		latexcad.exe & Windows executable & usually c:\bs tex \\
		latexcad.hlp & Windows help file  & same as above  \\
		vbrun300.dll & dynamic link library    & c:\bs windows \bs system \\
		cmdialog.vbx & common dialog control   & c:\bs windows \bs system \\
		commdlg.dll  & common dialog library   & c:\bs windows \bs system \\
		latexcad.sty & macro file         & in TEXINPUTS path \\
		fancybox.sty & box macros         & as above \\
		epic.sty     & picture macros     & as above \\
		eepic.sty    & extended picture macros & as above \\
		epsf.sty     & postscript macros  & as above \\
		rotate.sty   & postscript text rotation macros  & as above \\
		\hline
	\end{tabular}
	\caption{Contents of the distribution and required locations.}
	\label{tab:files}
\end{table}

%%%%%% operation 
\section{ Operation }

%%%%% CAD operations
\subsection{ CAD Operations }
Upon entry to LaTeX-CAD, the user is placed in edit mode. Objects in
the categories  Lines, Boxes, Curves, Dots or Text may be drawn via
the appropriate menu selection. When selected, the `action' box
reflects the appropriate action to be taken (for example, selecting
the end point of a line, or the bounding box of a circle). Three
line thicknesses may be selected, and the 2mm grid may be used in
`snap to' mode, in which all selections are rounded to the nearest
grid point. A number of objects incorporate a textual component, in
which case the user is prompted via a standard dialog box for the
text. This text may be any \LaTeX\/ text, so constructs such as
\verb! $\sum f(x)$! and \verb! \textit{\small current level}!
are valid. Several commonly-used objects may be selected via
keyboard shortcuts (such as ctrl-r for vectors). These are shown
on the pull-down menus.

When drawing an object, the current operation is cancelled by clicking
the \textit{right} mouse button, or pressing the escape key. Lines (solid,
dashed and dotted) and vectors (arrows) are a slight exception to this.
When drawing a line, the next line is continued by default from the
end of the previous one. Clicking the right mouse will cancel this follow-on,
\textit{but you will remain in line-draw mode}. A second click on the right mouse
button is necessary to cancel line-drawing mode. 

\textit{Single objects} may then be selected by clicking the left mouse button when
the mouse cursor is at approximately the centre of the object (exception:
text objects which are aligned to the left or right).
This object thus selected will appear in a different color, and it is
possible to delete or cut the object using the menu (or shortcut keys).
If the object contains text, the `edit text' button will be enabled.
In the case of objects which are very close together, additional left-mouse
clicks will select the next nearest object in sequence.

To move a \textit{single object}, hold down the \textit{control} key and then
select the object by clicking with the left mouse button. The mouse
cursor will change to a different style of pointer, and an outline of the
object will appear. Simply move the cursor to the desired object location
and click the mouse to place the object in a new position.

\textit{Groups of objects} may be selected by holding down the control key, and
dragging the mouse (ie hold down mouse button and move mouse). A moving
selection rectangle will appear. Objects within this selection region will
then change color.
The selected objects are then in the copy buffer, and may then be cut or
pasted according to the edit menu. Shortcuts for these operations are
\textit{ctrl-x} (cut), \textit{ctrl-v} (paste) and \textit{del} (delete).

Data points outlining a curve or a GnuPlot picture may also be
imported into LaTeX-CAD. This operation is covered in
Section~\ref{sec:xygnu}.

The current drawing is saved under the file menu, and may be reloaded
later. A recent file list is also maintained. The default extensions
are .lp (\LaTeX\/ picture), .sp (Sli\TeX\/ picture), .gp (GnuPlot using
eepic macros, see later). These extensions are simply for convenience
in identifying file types, and are not used in any way by LaTeX-CAD.
Often, for example, a similar picture is required for Sli\TeX\ using
thicker lines than for a printed work, thus the .sp extension may
be used.

A configuration file, latexcad.ini, is maintained in the Windows
directory. The format of this file is shown in Figure~\ref{fig:inifile}.

\begin{figure}
	\begin{center}
	\begin{tabular}{l}
		\hline
		$[$ DRAWING $]$ \\
		Lines=thin \\
		SnapToGrid=on \\
		DrawGrid=on \\
		Crosshair=off\\
		NormalizeOnSave=no \\
		ScreenFontSize=9.6\\
		\\
		$[$WINDOW$]$ \\
		WindowState=FullScreen \\
		Width= 12384 \\
		Height= 9312 \\
		\\
		$[$RECENT$]$ \\
		File 1=d:\bs latex\bs latexcad\bs gnuplot.lp \\
		File 2=d:\bs latex\bs latexcad\bs lcadeg1.lp \\
		File 3=d:\bs latex\bs latexcad\bs lcadeg.lp \\
		File 4=c:\bs temp\bs temp.lp \\
		\hline\hline
	\end{tabular}
	\caption{latexcad.ini file}
	\label{fig:inifile}
	\end{center}
\end{figure}


%%% latex source
\subsection{ \LaTeX\/ Source } \label{sec:source}
Once the drawing has been created and saved, it may be incorporated
into the \LaTeX\/ source file using either normal \LaTeXe\ conventions
as shown in Figure~\ref{fig:longhand}, or a number of supplied 
``shorthand'' macros. Note that the ``long'' method is required if you 
wish to use, for example, long and short captions or the ``caption''
package, or you wish greater control over the final appearance of the drawing.

\begin{figure}
	\centering
	\begin{tabbing}{ll}
		\hspace{3mm} \= \kill
		 \verb! %-------- include LaTeX-CAD drawing test.lp ------------- ! \\
		 \verb! \setlength{\unitlength}{1.0mm}  % use 1.0mm default scale ! \\
		 \verb! \begin{figure} ! \\
			\> \verb! \centering !  \\
			\> \verb! \input{test.lp}   % output file from latexcad ! \\
			\> \verb! \caption[short caption here]{long caption is placed here} ! \\
			\> \verb! \label{fig:test}  % NOTE: \caption MUST come before the \label ! \\
		 \verb! \end{figure} !   \\
		 \verb! %-------------------------------------------------------- ! \\
	\end{tabbing}
	\caption{Incorporating LaTeX-CAD drawings -- longhand method.}
	\label{fig:longhand}
\end{figure}

The usage of these macros is
\begin{verbatim}
	\placedrawing{drawfilename}{drawcaption}{drawlabel}
\end{verbatim}

Where \textit{drawfilename} is the name of the file saved in LaTeX-CAD,
\textit{drawcaption} is the caption placed beneath the figure,
\textit{drawlabel} is the crossreferencing label. For example,
Figure~\ref{fig:lcadeg} exists in a file \textit{lcadeg.lp}, and is
placed in the .tex file by the command
\begin{verbatim}
\placedrawing{lcadeg.lp}{\LaTeX-CAD standard shapes}{fig:lcadeg}
\end{verbatim}
and is cross-referenced as
\begin{verbatim}
	 ... see Figure~\ref{fig:lcadeg} ....
\end{verbatim}

The macros are summarized in Figure~\ref{fig:lcadmacros}. Note that
the PostScript macros are not required by LaTeX-CAD, but are included
for reference.


\begin{figure}
	\begin{center}
	\begin{tabular}{l}
		\hline
		\verb! \placedrawing{test.lp}{A Diagram.}{fig:test}! \\
			\emph{Place a LaTeX-CAD drawing} \\
			\verb! \drawingscale{0.4mm}! \\
			\emph{Alter the default 1.0mm scaling for subsequent drawings} \\ 
			\verb! \placedrawing[p]{test.lp}{A Diagram.}{fig:test}! \\
			\emph{Place a drawing, with placement argument \underline{p}} \\
			\verb! \placedrawing*{test.lp}! \\
			\emph{Place a LaTeX-CAD drawing, with no `Figure' label} \\
			\verb! \placegraph{test.gp}{A test graph.}{fig:test}! \\
			\emph{Place a GnuPlot (eepic macro) graph} \\
			\verb! \placegraph[p]{test.gp}{A test graph.}{fig:test}! \\
			\emph{As above, with placement argument \underline{p}} \\
			\verb! \placegraph*{test.gp}!   \\
			\emph{Place a GnuPlot graph, with no `Figure' caption} \\
			\verb! \placepostscript{test.eps}{A test.}{fig:test}! \\
			\emph{Place an encapsulatedPostScript file} \\
			\verb! \setpostscriptwidth{80mm}!       \\
			\emph{Change the default 140mm width for next EPS figure }\\
			\verb! \placepostscript[p]{test.eps}{A test.}{fig:test}! \\
			\emph{Place a PostScript figure with positioning arguments} \\
			\verb! \placepostscript*{test.eps}! \\
			\emph{Place a PostScript figure with no `Figure' caption} \\
			\verb! \placepostscript*[p]{test.eps}! \\
			\emph{A combination of the above} \\
			\verb! \placeslidedrawing{test.sp}{ A Slide Drawing }! \\
			\emph{Place a LaTeX-CAD drawing on a slide} \\
			\verb! \placeslidegraph{test.gp}{ A Slide Graph }! \\
			\emph{Place a GnuPlot graph on a slide} \\ 
			\verb! \placeslidepostscript{test.eps}{Slide EPS }! \\
			\emph{Place an encapsulated PostScript figure on a slide} \\
		\hline\hline
	\end{tabular}
	\end{center}
	\caption{LaTeX-CAD macros}
	\label{fig:lcadmacros}
\end{figure}

The drawing file may be manually edited, but care should be taken
to ensure only constructs known to LaTeX-CAD are incorporated.
Figure~\ref{fig:lcadfile} shows an example LaTeX-CAD file.


\begin{figure}
	\begin{center}
	\begin{tabular}{l}
		\hline
		\% LaTeX-CAD 1.9 - requires latexcad.sty \\
		\bs begin\{picture\}(112,80)             \\
		\bs thinlines                            \\
		\bs drawdotline\{8.07\}\{13.03\}\{8.07\}\{70.55\} \\
		\bs thicklines                          \\
		\bs drawdotline\{8.07\}\{13.03\}\{8.07\}\{70.55\} \\
		\bs drawrighttext\{6.31\}\{13.03\}\textit\{some text\} \\
		\bs end\{picture\}                      \\
		\hline\hline
	\end{tabular}
	\caption{LaTeX-CAD output file}
	\label{fig:lcadfile}
	\end{center}
\end{figure}
		
%%%% Object types
\subsection{ Object Types } \label{sec:objecttypes}
Several object types are available in LaTeX-CAD. These are
summarized below.

\begin{description}
	\item[Lines] Standard lines to any point, dashed lines,
			dotted lines and vectors. Note that
			vectors are not available in all possible
			slopes.
	\item[Boxes] Framed boxed of any dimension with optional
			interior text, dashed-line boxes, and
			boxes with a `shadow'.
	\item[Curves] Circles with optional centered text,
			ovals, ellipses, Bezier curves and arcs.
			Arcs may be drawn by either selecting the
			end points, or selecting the center and
			radius. The direction of the arc is determined
			by the order of endpoint selection.
	\item[Dots] Small dots, thicker dots and outline dots are available.
	\item[Braces] Braces with orientation to the left, right,
			above or below.
	\item[Text] Any standard \LaTeX\ text may be placed with the
			reference point to the left, center or
			right of the (invisible) box enclosing the
			text. Text may also be enclosed in an outlined box,
			a double-outline box, a shadowed box, or an oval.
			Text may be rotated $\pm 90^\circ $, if a
			PostScript output device is used.
			(On-screen, rotated text will \emph{not} appear rotated).
			Stacked text (verically aligned)
			is also available using the \textit{shortstack} command.
			Separate each line using `\bs\bs'. Note that the
			size of \textit{text} on the screen under LaTeX-CAD will only
			ever be an approximation of the true output dvi/ps
			size. This is because the true \LaTeX\/ font sizes are
			unknown to LaTeX-CAD. A default font size of 9.6 points
			is used - if this seems unrealistic with your screen
			drivers, change the `ScreenFontSize' parameter in
			\textit{latexcad.ini} .
			
	\item[Arbitrary Paths] Any arbitrary path may be created by
			means of the XY data point facility. This is
			covered in Section~\ref{sec:xygnu}.
	\item[PostScript Graphic] PostScript pictures may be placed in a
			\LaTeX-CAD drawing. On the screen, they will
			appear as a colored rectangle.
	\item[Shading] A rectangular shading region may be selected, with shading
			from 0\% (white) to 100\% (black). Shaded circles are also
			available. Note that the shaded circles also draw the
			outline, whereas the shaded boxes do not. If a solid outline
			is required for a shaded box it must be drawn separately.
			Note that the LaTeXCAD shading appearance is consistent
			with a PostScript printer; some \textsf{dvi} previewers
			appear to handle the shading/obscuring slightly differently.
\end{description}

Three line thicknesses are available for objects. The `thinlines'
thickness is recommended for standard documents, whilst the `thicklines'
thickness is recommended for Sli\TeX\ slides. These are illustrated in
Figure~\ref{fig:linethick}.

\placedrawing{linethic.lp}{Line thicknesses}{fig:linethick}


\subsection{ X-Y Data and GnuPlots }
\label{sec:xygnu}
Arbitrary paths may be constructed using the x-y import facility.
This is useful for importing waveforms and the like to illustrate
diagrams, where the waveform shape is more easily generated as
a series of $(x,y)$ points. 

When drawing x-y points, a bounding box is drawn to enclose the
points. The points are then read in, and plotted in this bounding
box. The points are read in from an ASCII text file, one point
per line, separated by whitespace or commas. An example of a
QBASIC program to generate such data points is shown in
Figure~\ref{fig:plotegsrc}, with the result shown in Figure~\ref{fig:lcadeg}.
(The `lgrind' package is used here for source code formatting.)

\lagrind[h]{ploteg.lg}{BASIC data file generation}{fig:plotegsrc}

Plots output from the GnuPlot plotting package may also be
incorporated into LaTeX-CAD, for editing. This is illustrated in
Figure~\ref{fig:gnuplot}. Note that the steps shown below
\textit{must} be taken before a plot file suitable for LaTeX-CAD is
produced:
\begin{center}
\begin{tabular}{l}
	set term eepic  \\
	set out 'plotfile.gp'  \\
	replot  \\
	set out \\
	set term windows \\
\end{tabular}
\end{center}
	
Since LaTeX-CAD has a finite object space, some very large plots
may be too large to incorporate. In this case, it is recommended
that the \textit{set samples} command be used to decrease the sampling
grid used by Gnuplot. In the case of 3-D plots, the corresponding
command is \textit{set isosamples}. In either case, plots will render
more quickly if the number of samples is reduced.
	

%%%%% changes from previous versions
\subsection{Changes from Previous Versions}
A number of enhancements and bug fixes have been introduced
since version 1.0. Briefly, these are :-
\begin{itemize}
	\item Changed name of \LaTeX\/ style file to `latexcad.sty' to
		avoid conflicts at some installations.
	\item Changes to the user macros for placing objects, to make them
		more `\LaTeX-consistent'. (See documentation later
		in Section~\ref{sec:source}).
	\item Changed on-screen text font size to more closely
		\textit{approximate} the true font size in the output file.
	\item Optional drawing crosshair. Useful for aligning objects.
	\item Editable object text.
	\item Changed object selection/movement method.  Click to
		edit an object's text, control-click to move an object,
		control-drag to select a region. Movement of selected
		object to the copy buffer is now automatic (ie don't
		need to explicitly select \textit{edit-copy} from the menu).
	\item Several new shortcut control keystrokes have been added.
		See the pull-down menu for these.
	\item Shading of a square region is possible, with shading from
		0 to 100\%. This applies to PostScript devices, and \textsf{dvi}
		drivers that support this particular \textit{tpic \bs special}.
		Note that the LaTeXCAD shading appearance is consistent
		with a PostScript printer; some \textsf{dvi} previewers appear to
		handle the shading/obscuring slightly differently
		(ie visibility of objects behind a shaded box).
	\item Ability to import a PostScript graphic into the LaTeX-CAD drawing.
	\item Stacked text option (useful for alignment of labels etc).
	\item Vertical text going up or down the page (PostScript only).
	\item Fields to explain the meaning of text options in text
		object dialog box.
	\item (Version 1.2) Added horizontal/vertical scroll bars (by popular request!)
	\item Fixed .ini startup bug.
	\item Changed menu/tools to (hopefully) fit on smaller screens.
	\item Normalization on exit is now optional, and remembered in .ini file.
	\item (Version 1.3) Vertical and Horizontal scroll bars.
	\item Changed default .ini file format.
	\item Re-arrange menu/tools for smaller screen resolutions.
	\item Added four-point Bezier curve.
	\item Fixed bug in vector objects when `snap to grid' is off.
	\item The current drawing object type is maintained after each draw.
	\item Changed on-screen text fonts to proportional.
	\item Added font size to .ini file.
	\item Added shaded circle object.
	\item Added framed text object.
	\item PostScript and DVI documentation no longer included so as to reduce
		distribution size (these must be compiled; sources are included)
	\item Made default screen size larger (180mm across).
	\item (Version 1.4) Fixed some screen font problems.
	\item Added help file.
	\item (Version 1.5) Changed single-object selection to allow circular buffer.
		(for overlayed objects)
	\item Continuation for line-type objects from endpoint of previous line is now automatic.
	\item Bound F1 to help function.
	\item Fixed offset bug in shaded boxes.
	\item Fixed `overflow' bug under certain startup conditions.
	\item Added optional end arrows to Bezier curves.
	\item Version 1.6 - fixed startup bug when running under
		Windows'95 (does not affect Windows 3.x). Thanks to
		Michael Basler for beta-testing this patch.
	\item Line/vector drawing mode is continued until two right mouse-clicks.
		(This speeds up a lot of drawing construction).
	\item (Version 1.7) - Alignment problem with shadow boxes (executable fix),
		Alignment problem with shaded boxes (style file fix).
	\item (Version 1.8) - Re-bundled with correct epic/eepic.
	\item Added example to documentation about incorporating drawings
			without using special macros (only default \LaTeX\/ constructs)
	\item (Version 1.9) - Re-bundled with modified epic/eepic 
		(cr/lf problem under some \LaTeX's).
\end{itemize}
Thanks to all those who contributed suggestions and reports of 
problems in earlier versions. 


%%%% Acknowledgements.
\section{ Acknowledgements }
The following \LaTeX\/ macro packages are incorporated as part of
LaTeX-CAD :-
\begin{description}
	\item[epic] Picture macros (Sunil Podar)
	\item[eepic] Additional picture macros (Conrad Kwok)
	\item[fancybox] Box outlines (Timothy Van Zandt)
	\item[epsf] Postscript macros (Tomas Rokicki)
	\item[rotate] PostScript text-rotation macros (Tomas Rokicki)
\end{description}

The \textit{lgrind} package for source code formatting
(George Reilly, Jerry Leichter, Michael Piefel \textit{et al})
as used to typeset the documentation, but is not part of LaTeXCAD itself.
LaTeX-CAD naturally depends on Donald Knuth's \TeX\ and
Leslie Lamport's \LaTeX\/, plus of course the \LaTeXe\ extensions. 
Inspiration for LaTeX-CAD was taken from the emtex distribution's texcad. 
LaTeX-CAD is written in Visual Basic 3.0 for Windows. 

The author welcomes any constructive suggestions for improvement,
at the email address cited.

\end{document}

